%----------------------------------------------------------------------------------------
%	Salih Karabulut - CV
%----------------------------------------------------------------------------------------

%----------------------------------------------------------------------------------------
%	PACKAGES AND OTHER DOCUMENT CONFIGURATIONS
%----------------------------------------------------------------------------------------

\documentclass[a4paper,12pt]{memoir} % Font and paper size

\input{structure.tex} % Include the file specifying document layout and packages

%----------------------------------------------------------------------------------------
%	NAME AND CONTACT INFORMATION 
%----------------------------------------------------------------------------------------

\userinformation{ % Set the content that goes into the sidebar of each page
\begin{flushright}
% Comment out this figure block if you don't want a photo
%\includegraphics[width=0.6\columnwidth]{_.jpeg}\\[\baselineskip] % Your photo
\small % Smaller font size
Salih Karabulut \\ % Your name
\url{salih14k@gmail.com} \\ % Your URL
\url{karabulut18@itu.edu.tr} \\ % Your email address
(506) 601 8653 \\ % Your phone number
\Sep % Some whitespace

\textbf{\href{https://github.com/karabulut18}{Github}} \\
\Sep % Some whitespace

\textbf{\href{https://www.linkedin.com/in/salih-karabulut-4b78271b5}{LinkedIn}} \\
\Sep % Some whitespace

\Sep % Some whitespace
\textbf{Address} \\
Kağıthane, İstanbul\\ % Address 2
Turkey \\ % Address 3
\vfill % Whitespace under this block to push it up under the photo
\end{flushright}
}

%----------------------------------------------------------------------------------------

\begin{document}

\userinformation % Print your information in the left column

\framebreak % End of the first column

%----------------------------------------------------------------------------------------
%	HEADING
%----------------------------------------------------------------------------------------

\cvheading{Salih Karabulut} % Large heading - your name

\cvsubheading{Computer Engineer} % Subheading - your occupation/specialization

%----------------------------------------------------------------------------------------
%	EXPERIENCE
%----------------------------------------------------------------------------------------
\CVSection{Experience}
{\small

\CVItem{Software Developer, HFT\footnote{Company name withheld due to confidentiality} \hfill 05/23 -- 07/25}{
\begin{itemize}
\item Architected a \textbf{distributed task-allocation system} in C++ over TCP using a requester-server-processor topology, ensuring fault-tolerant delivery and real--time execution.
\item Engineered a high-throughput \textbf{TCP file server} optimized for concurrent read/write operations; implemented a custom static library and a pub/sub mechanism for automated vertical file synchronization.
\item Developed a \textbf{low-latency protocol tunneling application} (TCP $\leftrightarrow$ UDP) for seamless bi-directional translation between legacy and modern systems.
\item Maintained and optimized mission-critical \textbf{C/C++ shared libraries}, delivering robust tooling for logging, config parsing, and high-performance file IO.
\item Deployed a \textbf{scalable state-monitoring solution} for a fleet of 100+ machines via TCP, enabling real-time infrastructure visibility and health tracking.
\item Developed a \textbf{real-time state-synchronization system} designed for remote environment control, focusing on high-performance state propagation and system responsiveness.
\end{itemize}
}

\CVItem{Blockchain Developer Intern, Doğuş Technology \hfill 05/22 -- 07/22}{
\begin{itemize}
\item Prototyped a blockchain e-commerce platform using smart contracts and MetaMask.
\end{itemize}
}
\CVItem{Android Developer Intern, IoT Turkey \hfill 06/21 -- 08/21}{
\begin{itemize}
\item Architected Android applications using \textbf{Kotlin} and \textbf{Firebase}, focusing on real-time data synchronization.
\end{itemize}
}

\CVItem{Volunteer Experience}{
\begin{itemize}
\item \textbf{ITU Volunteer Club} (2018--2019): Tutored high school students in \textbf{Geometry} and Math, fostering equal opportunity in education.
\item \textbf{IoT Turkey}: Contributed to business development and strategic partnerships.
\end{itemize}
}
}

%----------------------------------------------------------------------------------------
%	EDUCATION
%----------------------------------------------------------------------------------------
\CVSection{Education}

{\small
\CVItem{Istanbul Technical University, B.S. Computer Engineering \hfill Graduated 2025}{
\begin{itemize}
\item \textbf{Graduation Project:} Developed a modular testbed for \href{https://github.com/karabulut18/semanticCommunication}{semantic communication} using GNU Radio and Docker, simulating wireless channel impairments and BPSK modulation.
\end{itemize}
}
}

%----------------------------------------------------------------------------------------
%	PROJECTS
%----------------------------------------------------------------------------------------
\begin{minipage}{\linewidth}
\CVSection{Projects}

{\small
\CVItem{\href{https://github.com/karabulut18/redis}{High-Performance Redis Clone} \hfill \textit{C++17, POSIX, TCP Sockets}}
{
\begin{itemize}
\item Architected a multi-threaded, high-throughput Key-Value store with \textbf{zero-copy network buffers} and custom \textbf{RESP3 protocol} parsing.
\item Engineered a \textbf{lock-free concurrent Pub/Sub engine} utilizing Producer-Consumer ring buffers to eliminate mutex contention across I/O and worker threads.
\item Implemented robust persistence mechanics including point-in-time \textbf{RDB binary snapshotting} and \textbf{AOF background rewrites} leveraging Linux \texttt{fork()} Copy-On-Write semantics.
\end{itemize}
}
}
\end{minipage}

\begin{minipage}{\linewidth}
\CVItem{\href{https://github.com/karabulut18/systemProgramming}{System Programming Experiments} \hfill \textit{C, POSIX}}
{
\begin{itemize}
\item Implemented various low-level utilities exploring process management, concurrency, and POSIX system calls.
\end{itemize}
}
\end{minipage}

\begin{minipage}{\linewidth}
\CVItem{\href{https://github.com/karabulut18/carpik_kaldirimlar}{Çarpık Kaldırımlar} | \href{https://carpik-kaldirimlar.web.app/}{Live Web App} \hfill \textit{Flutter, Dart, Firebase}}
{
\begin{itemize}
\item Developed a scalable web application utilizing \textbf{Flutter} and \textbf{Firebase} (Firestore, Auth), featuring rich-text Markdown editing and link previews.
\item Engineered a \textbf{relational social engine} supporting nested threaded replies, user tagging, and real-time like synchronizations.
\item Implemented robust server-side Firestore security rules and a role-based Admin Panel for secure community moderation.
\end{itemize}
}
\end{minipage}

%----------------------------------------------------------------------------------------
%	SKILLS
%----------------------------------------------------------------------------------------
\begin{minipage}{\linewidth}
\CVSection{Technical Skills}
{\small
\begin{tabular}{@{}p{0.31\linewidth} p{0.31\linewidth} p{0.31\linewidth}@{}}
\textbf{Languages} & \textbf{Tools} & \textbf{Topics} \\
\bluebullet C/C++, Python & \bluebullet Git, Docker, Linux & \bluebullet Distributed Sys. \\
\bluebullet Kotlin & \bluebullet GDB, Valgrind, SVN & \bluebullet Network Prog. \\
& \bluebullet GNU Radio, CMake & \bluebullet Multithreading \\
\end{tabular}
}
\end{minipage}



%\Sep % Extra whitespace after the end of a major section
%	NEW PAGE DELIMITER
%	Place this block wherever you would like the content of your CV to go onto the next page
%----------------------------------------------------------------------------------------
 % Start a new page
 % End of the first column

%----------------------------------------------------------------------------------------
%	AWARDS
%---------------------------------------------------------------------------------------------------------------------------------------------------------------
%	INTERESTS
%----------------------------------------------------------------------------------------

\end{document}